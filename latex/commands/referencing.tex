% !TEX root = ../ecc.tex

\newcommand{\chp}[2]{
    \providecommand{\curchp}{}
    \renewcommand{\curchp}{#2}
    \chapter{#1}
    \label{chp:#2}
    \input{contents/#2.tex}
}

\renewcommand{\sec}[2]{
    \section{#1}
    \label{sec:#2}
    \input{contents/\curchp/#2.tex}
}

\newcommand{\sbs}[2]{
    \subsection{#1}
    \label{sbs:#2}
}

\newcommand{\sss}[2]{
    \subsubsection{#1}
    \label{sss:#2}
}

\newenvironment{fig}[2]{
    \begin{figure}[H]
    \newcommand{\figcaption}{#1}
    \newcommand{\figlabel}{fig:#2}
}{
    \caption{\figcaption}
    \label{\figlabel}
    \end{figure}
}

\newenvironment{tab}[2]{
    \begin{table}[H]
    \newcommand{\tabcaption}{#1}
    \newcommand{\tablabel}{fig:#2}
}{
    \caption{\tabcaption}
    \label{\tablabel}
    \end{table}
}

\newenvironment{thm}[2]{
    \begin{theorem}[#1]
    \label{thm:#2}
}{
    \end{theorem}
}

\newenvironment{lem}[2]{
    \begin{lemma}[#1]
    \label{lem:#2}
}{
    \end{lemma}
}

\newenvironment{cor}[2]{
    \begin{corollary}[#1]
    \label{cor:#2}
}{
    \end{corollary}
}

\newenvironment{dfn}[2]{
    \begin{definition}[#1]
    \label{dfn:#2}
}{
    \end{definition}
}

\newenvironment{exm}[2]{
    \begin{example}[#1]
    \label{exm:#2}
}{
    \end{example}
}

\newenvironment{rmk}[2]{
    \begin{remark}[#1]
    \label{rmk:#2}
}{
    \end{remark}
}

\newenvironment{alg}[2]{
    \begin{algorithm}
    \caption{#1}
    \label{alg:#2}
}{
    \end{algorithm}
}

\crefformat{equation}{(#2#1#3)}
\newenvironment{eqn}[1]{
    \align
    \notag
    \ifthenelse{\equal{#1}{}}{}{\label{eqn:#1}}
}{
    \endalign
}

\crefformat{enumi}{(#2#1#3)}

% %\addto{\captionsafrikaans}{\renewcommand{\bibname}{Lys van Verwysings}}
% %\addto{\captionsenglish}{\renewcommand{\bibname}{List of References}}
