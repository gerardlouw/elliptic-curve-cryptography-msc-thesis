In this appendix, we give Sage code for performing computations in the endomorphism ring of an elliptic curve $E / \Fq$ modulo the division polynomial $\psi_\ell$, which is used in Schoof's algorithm.

\begin{sagecode}
class EndomorphismMod:
    def __init__(self, E, fx, fy, psi):
        R.<x> = E.base_field()['x'].quotient(psi)
        self.E = E
        self.fx = R(fx)
        self.fy = R(fy)
        self.psi = psi
        self.f = x^3 + E.a4() * x + E.a6()
        self.A = E.a4()

    def __add__(self, other):
        if isinstance(other, Integer):
            other *= IdentityEndomorphismMod(self.E, self.psi)
        if isinstance(other, ZeroEndomorphismMod):
            return self
        if self.fx == other.fx:
            if self.fy == other.fy:
                m = (3 * self.fx^2 + self.A) / (2 * self.fy * self.f)
            else:
                return ZeroEndomorphismMod(self.E, self.psi)
        else:
            div, inv, _ = xgcd((self.fx - other.fx).lift(), self.psi)
            if div.is_constant():
                m = (self.fy - other.fy) * inv
            else:
                raise ZeroDivisionError(div)
        fx = m^2 * self.f - self.fx - other.fx
        fy = m * (self.fx - fx) - self.fy
        return EndomorphismMod(self.E, fx, fy, self.psi)

    def __radd__(self, other):
        return self + other

    def __neg__(self):
        return EndomorphismMod(self.E, self.fx, -self.fy, self.psi)

    def __sub__(self, other):
        return self + -other

    def __rsub__(self, other):
        return -(self + -other)

    def __mul__(self, other):
        if isinstance(other, Integer):
            if other < 0:
                return -self * -other
            alpha = ZeroEndomorphismMod(self.E, self.psi)
            while other > 0:
                if other % 2 == 1:
                    alpha += self
                self += self
                other //= 2
            return alpha
        fx = self.fx.lift()(other.fx)
        fy = self.fy.lift()(other.fx) * other.fy
        return EndomorphismMod(self.E, fx, fy, self.psi)

    def __rmul__(self, other):
        return self * other

    def __pow__(self, n):
        alpha = IdentityEndomorphismMod(self.E, self.psi)
        while n > 0:
            if n % 2 == 1:
                alpha *= self
            self *= self
            n //= 2
        return alpha

    def __eq__(self, other):
        if isinstance(other, Integer):
            other *= IdentityEndomorphismMod(self.E, self.psi)
        if isinstance(other, ZeroEndomorphismMod):
            return False
        return self.fx == other.fx and self.fy == other.fy
\end{sagecode}

\begin{sagecode}
class ZeroEndomorphismMod(EndomorphismMod):
    def __init__(self, E, psi):
        self.E = E
        self.psi = psi

    def __neg__(self):
        return self

    def __add__(self, other):
        return other

    def __mul__(self, other):
        return self

    def __eq__(self, other):
        if isinstance(other, Integer):
            other *= EndomorphismMod(self.E, x, 1, self.psi)
        return isinstance(other, ZeroEndomorphismMod)
\end{sagecode}

\begin{sagecode}
class IdentityEndomorphismMod(EndomorphismMod):
    def __init__(self, E, psi):
        _.<x> = E.base_field()['x'].quotient(psi)
        EndomorphismMod.__init__(self, E, x, 1, psi)
\end{sagecode}

\begin{sagecode}
class FrobeniusEndomorphismMod(EndomorphismMod):
    def __init__(self, E, psi):
        q = E.base_field().order()
        _.<x> = E.base_field()['x'].quotient(psi)
        f = x^3 + E.a4()*x + E.a6()
        EndomorphismMod.__init__(self, E, x^q, f^((q - 1) // 2), psi)
\end{sagecode}
