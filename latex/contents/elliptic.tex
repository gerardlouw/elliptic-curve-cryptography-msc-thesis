% !TEX root = ../ecc.tex

%TODO define smoothness and genus
%TODO state Riemann-Roch, perhaps explain Weierstrass form isomorphism in more detail

%XXX more general: http://en.wikipedia.org/wiki/Genus%E2%80%93degree_formula

%TODO ease into definition

%An \emph{elliptic curve} over $K$ is a smooth, projective curve of genus one over $K$ with a distinguished $K$-rational point, called the \emph{base point} \citep{Silverman}.

%TODO discriminant. j-invariant. invariant differential?

In this chapter, we cover the background on elliptic curves needed throughout the rest of the thesis. Our exposition borrows heavily from the books \citep{Silverman,Washington}, as well as the lecture notes \citep{Sutherland}.

%\begin{dfn}{}{}
%An \emph{elliptic curve} over $K$ is the projective closure of a smooth affine plane curve
%\begin{eqn}{}
%E : y^2 + a_1 xy + a_3 y = x^3 + a_2 x^2 + a_4 x + a_6
%\end{eqn}
%over $K$.
%\end{dfn}

%\chp{Elliptic and hyperelliptic curves}

%\sec{Elliptic curves}




%TODO weil pairing, det \psi_\ell = deg \psi



%\begin{lem}{}{}
%The degree of a non-zero morphism $\alpha \defeq (\alpha_{x; 1} / \alpha_{x; 2}, \alpha_y y) \in Mor(E, E')$, where $\alpha_{x; 1}, \alpha_{x; 2} \in K[x]$ are relatively prime and $\alpha_y \in K(x)$ is given by $\deg \alpha = \max(\deg \alpha_{x; 1}, \deg \alpha_{x; 2})$.

%TODO torsion subgroups
%TODO endomorphisms/isogenies, characteristic polynomial of frobenius endomorphism
%TODO hasse's theorem
%TODO can do edwards curves, they are smooth completions of smooth affine curves, there are two points at infinity whose blow-ups are four points defined over k(\sqrt(d)), where d is assumed to be a quadratic non-residue

%TODO Every smooth affine curve over a perfect field $K$ is birationally equivalent to a unique (up to isomorphism) smooth projective curve over $K$, which is called its \emph{smooth completion}.


%\begin{dfn}{}{}
%The \emph{$j$-invariant} of an elliptic curve in short Weierstrass form is the quantity
%\begin{eqn}{}
%j(E) \defeq
%\begin{cases}
%0 & \textnormal{if $\rchar(K) = 2$ and $a_1 = 0$} \\
%1 / a_6 & \textnormal{if $\rchar(K) = 2$ and $a_1 = 1$} \\
%%a_2^2 a_4^2 - a_2^3 a_6 - a_4^3 & \textnormal{if $\rchar(K) = 3$} \\
%1728 \frac{4a_4^3}{4 a_4^3 + 27a_6^2} & \textnormal{if $\rchar(K) \neq 2, 3$}.
%\end{cases}
%\end{eqn}

%Note that the $j$-invariant of an elliptic curve is well-defined, since the denominator in its formula is a constant multiple of the discriminant.

%TODO show j-invariant preserves kbar isomorphism
%TODO show that elliptic curves of each j-invariant arise


































\iffalse

\begin{dfn}{}{}
A \emph{hyperelliptic curve} of genus $g$ over $K$ is the smooth completion of a smooth affine plane curve
\begin{eqn}{}
H : y^2 + h y = f
\end{eqn}
over $K$, where $h \in K[x]$ is of degree at most $g$ and $f \in K[x]$ is monic of degree $2g + 1$. An \emph{elliptic curve} is a hyperelliptic curve of genus one.
\end{dfn}


It is clear that the line at infinity of $H$ consists of the single point $(0 : 1 : 0)$, which we denote $\ptinfty$. Furthermore, $\ptinfty$ is a singular point if and only if $g > 1$, so that elliptic curves are the only smooth hyperelliptic curves.


\begin{lem}{}{}
The polynomial $y^2 + h y - f$ is irreducible over $\al{K}$, so that $H$ is in fact a variety.
\end{lem}

%TODO discriminant, maybe only elliptic curves


If $\rchar(K) \neq 2$, then the isomorphism
\begin{eqn}{}
(x, y) \mapsto (x, y - h(x) / 2)
\end{eqn}
transforms $H$ to the simpler form
\begin{eqn}{}
H : y^2 = f,
\end{eqn}
where $f \in K[x]$ is monic of degree $2g + 1$.


The \emph{hyperelliptic involution} of $H$ is the automorphism
\begin{eqn}{}
\iota : (x, y) \mapsto (x, -y - h(x)).
\end{eqn}

\sbs{Regular and rational functions}

A regular function $r \in K[H]$ has a canonical representation of the form
\begin{eqn}{}
r = r_x + r_y y,
\end{eqn}
where $r_x, r_y \in K[x]$. To reduce a regular function $r$ to this form, we repeatedly replace a factor $y^2$ in every term of $r$ with such a factor by $f - hy$, until no such terms remain.

\fi

%The \emph{conjugate} of a regular function $r \defeq r_x + r_y y$ is the regular function $\overline{r} \defeq r \circ \iota = r_x - r_y(y + h)$

%The \emph{norm} of a regular function $r \in K[H]$ is the regular function $N(r) \defeq r \overline{r}$
