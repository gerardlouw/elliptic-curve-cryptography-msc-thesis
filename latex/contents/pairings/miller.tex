\begin{dfn}{}{miller_func}
Let $P \in E(L)[m]$ where $\rchar(K) \nmid m$. The \emph{$i$-th Miller function at $P$} is the rational function $f_{P;i} \in L(E)$ defined recursively by $f_{P;0} \defeq f_{P;1} \defeq 1$ and
\[
f_{P;i} \defeq f_{P;j} f_{P;i-j} \frac{\ell_{[j]P,[i-j]P}}{\ell_{[i]P,\ptinfty}}
\]
for $1 < i \leq m$ and $1 \leq j < i$, where $\ell_{P, Q}$ is the rational function in $L(E)$ corresponding to the line passing through $P$ and $Q \in E(L)[m]$ defined as
\[
\ell_{P, Q} \defeq \begin{cases}
1 & \text{if $P = Q = \ptinfty$} \\
x - x_Q & \text{if $P = \ptinfty$ and $Q \neq \ptinfty$} \\
x - x_P & \text{if $P \neq \ptinfty$ and $Q \in \set{\ptinfty, -Q}$} \\
y - (m_{P,Q} x + c_{P,Q}) & \text{otherwise},
\end{cases}
\]
where $m_{P,Q}$ and $c_{P,Q}$ are respectively the slope and the tangent of this line as in \cref{thm:group_structure}.
\end{dfn}

It remains to be shown that $f_{P;i}$ is a well-defined rational function, since $j$ may take on multiple values in the recursive definition. Since the rational function with a given divisor is unique up to multiplication by a constant, this can be seen from the following theorem by checking inductively that \[\left((x/y)^{-\ord_\ptinfty(f_{P;i})} f_{P;i}\right)(\ptinfty) = 1\]
for $i = 0, \ldots, m$, independant of the choices for $j$.

\begin{thm}{}{}
The $i$-th Miller function at a point $P \in E(L)[m]$ has the divisor $\div f_{P;i} = i(P) - ([i] P) - (i-1)(\ptinfty)$, and in particular $\div f_{P;m} = m(P) - m(\ptinfty)$.
\end{thm}
\begin{proof}
The result clearly holds for $f_{P; 0}$ and $f_{P; 1}$. Let $Q \in E(L)[m]$ and observe that \[\div \ell_{P, Q} = (P) + (Q) + (-(P + Q)) - 3(\ptinfty),\] and in particular \[\div \ell_{P, \ptinfty} = (P) + (-P) - 2(\ptinfty),\] so that \[\div \frac{\ell_{P, Q}}{\ell_{P+Q, \ptinfty}} = (P) + (Q) - (P + Q) - (\ptinfty).\] Assuming that the result holds for $f_{P;j}$ and $f_{P;i-j}$ where $1 \leq j < i$, we have
\begin{align*}
\div f_{P;i} =&\ (j+(i-j))(P) - ([j]P) - ([i-j]P) - (j+(i-j)-2)(\ptinfty) \\
&+ ([j]P) + ([i-j]P) - ([j+(i-j)]P) - (\ptinfty),
\end{align*}
so that \[\div f_{P;i} = i(P) - ([i]P) - (i-1)(\ptinfty)\] as required. The result follows by induction, where $\div f_{P;m} = m(P) - m(\ptinfty)$ since $[m] P = \ptinfty$.
\end{proof}

We now turn the problem of computing the $m$-th Miller function $f_{P;m}$. One can give a heuristic argument that the maximum degree of the polynomials in the canonical representation of $f_{P;i}$ is linear in $i$, so that directly computing the function $f_{P;m}$ would have at least linear time complexity in $m$. In practice, when working with an elliptic curve $E / \Fq$, the $m$-torsion subgroups we are interested in will have $m$ roughly as large as $q$, thus yielding such a computation infeasible for large $q$.

However, we may evaluate $f_{P;m}(Q)$ for a given $Q \in E(L)[m]$ more efficiently, by simply computing
\[f_{P;i}(Q) = f_{P;j}(Q) f_{P;i-j}(Q) \frac{\ell_{[j]P, [i-j]P}(Q)}{\ell_{[i]P,\ptinfty}(Q)}\] iteratively for some $1 \leq j < i$. Since $j$ may be chosen arbitrarily, this allows us to use a modified square-and-multiply algorithm, which yields a time complexity of $\Theta(\log m)$ operations in $E(L)$. This is known as \emph{Miller's algorithm}, due to the unpublished manuscript \citep{MillerPairing}. Sage code implementing Miller's algorithm is given in \cref{alg:miller}.

\begin{alg}{Miller}{miller}
\begin{sagecode}
def line(E, P, Q, R):
    """
    E: an elliptic curve
    (P, Q, R): points on E
    """
    if P == 0:
        if Q == 0:
            return 1
        else:
            return R[0] - Q[0]
    elif Q == 0 or P == -Q:
        return R[0] - P[0]
    elif P == Q:
        m = (3 * P[0]^2 + E.a4()) / (2 * P[1])
        c = (-P[0]^3 + E.a4() * P[0] + 2 * E.a6()) / (2 * P[1])
    else:
        m = (P[1] - Q[1]) / (P[0] - Q[0])
        c = (Q[0] * P[1] - P[0] * Q[1]) / (Q[0] - P[0])
    return R[1] - (m * R[0] + c)

def miller(E, m, P, Q):
    """
    E: an elliptic curve
    m: a positive integer
    (P, Q): m-torsion points on E
    """
    fPi, iP = 1, 0
    fPj, jP = 1, P
    while m > 0:
        if m % 2 == 1:
            fPi *= fPj * line(E, iP, jP, Q) / line(E, iP + jP, 0, Q)
            iP += jP
        fPj *= fPj * line(E, kP, kP, Q) / line(E, 2 * jP, 0, Q)
        jP += jP
        m //= 2
    return fPi
\end{sagecode}
\end{alg}

\begin{rmk}{}{miller_fail}
Miller's algorithm may fail to compute $f_{P;m}(Q)$ at a point $Q \in E(L)[m]$ if a denominator $\ell_{[k]P,\ptinfty}(Q) = 0$ is encountered during the computation for some intermediate $1 \leq k < m$. In this case, we gain the knowledge that $Q \in \langle P \rangle$, which will in fact help us to handle the situation in our applications.
\end{rmk}
