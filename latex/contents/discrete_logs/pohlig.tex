In the present section we describe a procedure known as the \emph{Pohlig-Hellman reduction}, first published in \citep{Pohlig}, which allows one to reduce the problem of solving a base-$G$ discrete logarithm to solving discrete logarithms in the prime-order subgroups of $\langle G \rangle$.

If $q_1 \ldots q_k$ is the prime factorization of $n$, where the $q_i \defeq \ell_i^{e_i}$ are powers of distinct primes, we proceed to solve the discrete logarithm $\log_G P$ for some $P \in \langle G \rangle$ as follows.

We proceed by computing the discrete logarithm $\log_{[n / q_i] G} [n / q_i] P$ in the subgroup $\langle [n / q_i] G \rangle$ of order $q_i$ of $\langle G \rangle$ for each $i = 1, \ldots, k$. Since this yields $\log_G P$ modulo $q_i$, the original discrete logarithm may then be reconstructed using the Chinese remainder theorem.

However, for some fixed factor $q = \ell^e$ in the prime factorization of $n$, the computation of the discrete logarithm $d \defeq \log_{[n / q] G} [n / q] P$ may be further reduced to computing a sequence of $e$ discrete logarithms in the $p$-order subgroup $\langle [n / \ell] G \rangle$ of $\langle G \rangle$.

First we represent the discrete logarithm $d$ in base $\ell$ as
\[
d = d_0 \ell^0 + \cdots + d_{e-1} \ell^{e-1}.
\]
Note that using this representation, we have \[d \equiv \sum_{j=0}^k d_j \ell^j \pmod{\ell^{k + 1}}\] for $k = 0, \ldots, e-1$, so that
\[
\log_{[n / \ell^{k+1}] G} [n / \ell^{k+1}] P = \sum_{j=0}^k d_j \ell^j.
\]

Thus to find $d_0$, we simply compute $\log_{[n / \ell] G} [n / \ell] P$. Assuming that $d_0, \ldots, d_{k-1}$ are already known, we may find $d_k$ by computing the discrete logarithm
\[
\log_{[n / \ell] G} \left( [n / \ell^{k+1}] P - \left[\sum_{j=0}^{k-1} d_j \ell^j\right] [n / \ell^{k+1}] G \right),
\]
which yields $d_k$ since
\begin{align*}
[n / \ell^{k+1}] P - \left[\sum_{j=0}^{k-1} d_j \ell^j\right] [n / \ell^{k+1}] G &= \left[\sum_{j=0}^{k} d_j \ell^j - \sum_{j=0}^{k-1} d_j \ell^j\right] [n / \ell^{k+1}] G \\
&= [d_k \ell^k] [n / \ell^{k+1}] G = [d_k] [n / \ell] G.
\end{align*}

Sage code for the Pohlig-Hellman reduction is given in \cref{alg:pohlig_hellman}. Note that to compute discrete logarithms in the prime-order subgroups of $\langle G \rangle$, this code uses the baby-step giant-step algorithm described in the previous section. Any other algorithm for solving discrete logarithms may be used in its place, including those which we will describe later in this chapter.

\begin{alg}{Pohlig-Hellman reduction}{pohlig_hellman}
\begin{sagecode}
def pohlig_hellman_reduction(G, P, n, n_factors):
    """
    G: a point
    P: a multiple of G
    n: the order of G
    n_factors: the prime factorisation of n
    """
    residues = []
    moduli = [l^e for l, e in n_factors]
    for l, e in n_factors:
        d = 0
        G_prime = (n // l) * G
        for j in (0..e-1):
            P_prime = (n // l^(j + 1)) * (P - d * G)
            d += l^j * baby_step_giant_step(G_prime, P_prime, l)
        residues.append(d)
    return crt(residues, moduli)
\end{sagecode}
\end{alg}

If $\ell$ is the largest prime factor of $n$, then the time complexity of the Pohlig-Hellman reduction is $O(\log n \sqrt{\ell})$ point additions, assuming that the baby-step giant-step algorithm is used for computing discrete logarithms in the prime-order subgroups of $\langle G \rangle$. To see this, note that
\[
\sum_{i=1}^k e_i \leq \log_2 n,
\]
and the largest discrete logarithm problem solved will be in a subgroup of order $\ell$.

Of course, this analysis assumes that the prime factorisation of $n$ is already known. In practice, this is a reasonable assumption, since the groups which are used for cryptography are typically proposed as standards, which do not change frequently. Furthermore, the \emph{general number field sieve} algorithm can factorise arbitrary integers with time complexity $L_n \lbrack 1/3, c \rbrack$, which is subexponential in $\log n$ and thus significantly more favourable than general techniques for discrete logarithms such as the baby-step giant-step algorithm \citep{Galbraith}.

\begin{rmk}{}{}
The existence of the Pohlig-Hellman reduction severely restricts which group varieties can be used for cryptographic purposes, since it means that solving the discrete logarithm problem in $\langle G \rangle$ is roughly as difficult as solving the discrete logarithm problem in its largest subgroup of prime order. For this reason, we will henceforth assume that the order of $G$ is a prime number $\ell$.
\end{rmk}

%TODO pollard's lambda

%TODO mention faster algorithms for classical DLP

%TODO mention divisor class groups of hyperelliptic curves

%TODO mention optimization for elliptic curves - we can reduce the collision search space to n / 2 by replacing a point with its negation if the y coordinate is odd
