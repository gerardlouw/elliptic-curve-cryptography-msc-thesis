Let $V / \Fq$ be a group variety with a fixed \emph{base point} $G \in V(\Fq)$ of order $n$ such that the addition and negation of points in $\langle G \rangle$ may be computed with time complexity polynomial in $\log n$. Some familiar examples of groups varieties which always have a base point with this property are the additive group $\Fq^+$, the multiplicative group $\un{\Fq}$, and an elliptic curve group $E / \Fq$.

Note that the map $\exp_G : k \mapsto [k] G$ is an isomorphism from $\Zmod{n}$ to $\langle G \rangle$ and may be evaluated at any element of $\Zmod{n}$ with time complexity polynomial in $\log n$ using a double-and-add algorithm. The problem of evaluating the inverse function $\log_G \defeq \exp_G^{-1}$ at some point in $\langle G \rangle$ is known as the \emph{discrete logarithm problem} -- in the literature, this name often refers to the special case for the group $\un{\Fq}$, while the case of an elliptic curve $E / \Fq$ is referred to as the \emph{elliptic curve discrete logarithm problem}.

It is clear that a brute force solution to the discrete logarithm problem exists, where one computes $[0] G, [1] G, \ldots$ until the desired point is found. This will require $\frac{n}{2}$ point additions on average, which leads to an average-case time complexity exponential in $\log n$. However, for the group variety $\Fq^+$, we may evaluate $\log_G P$ with time complexity polynomial in $\log n$ by performing the computation $G^{-1} P$ in $\Fq$. So-called \emph{index calculus algorithms} are a well-known family of algorithms which solve the discrete logarithm problem in $\un{\Fq}$ with a time complexity of $L_n \lbrack 1/3, c \rbrack$, which is subexponential in $\log n$.

The question arises for which groups varieties we may solve the discrete logarithm problem efficiently, say with a time complexity polynomial in $\log n$, and for which group varieties the discrete logarithm problem is hard, say only solvable with an expected time complexity exponential in $\log n$ \citep{hyperelliptic}.

In the present chapter, we first describe some algorithms for solving the discrete logarithm problem for general group varieties. These algorithms all have time complexity exponential in $\log n$ ($\Theta(\sqrt{n})$ to be precise). We then describe algorithms for solving the elliptic curve discrete logarithm problem efficiently for special classes of elliptic curves, which are to be avoided in cryptographic applications. In the next chapter, we proceed to define some cryptosystems whose security rely on the difficulty of the discrete logarithm problem.

There is a common variant of the discrete logarithm problem known as the \emph{Diffie-Hellman problem}, which is the problem of computing $\exp_G(\log_G P \log_G Q)$ for two points $P, Q \in \langle G \rangle$. Clearly an efficient solution to the discrete logarithm problem also gives an efficient solution to the Diffie-Hellman problem, but it is not known whether the converse is true for any group varieties. However, the general consensus is that the two problems are roughly equally difficult for the group varieties $\un{\Fq}$ and $E(\Fq)$.
