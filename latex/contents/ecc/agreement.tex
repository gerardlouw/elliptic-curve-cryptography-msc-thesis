Suppose that Alice and Bob wish to establish a common secret which is only known to them, while communicating only over a public network. A cryptosystem which allows one to do this is known as a \emph{key-agreement scheme}. A typical use case of key-agreement schemes is to establish a common secret which can be used as the \emph{symmetric key} for encrypting and decrypting subsequent communication using a \emph{symmetric cryptosystem} such as the \emph{Advanced Encryption Standard (AES)}. This is commonplace in practical applications, since the encryption and decryption procedures for symmetric cryptosystems are typically very fast to compute.

The \emph{Diffie-Hellman key-agreement scheme} was originally proposed by \citep{DiffieHellman}, using group varieties of the form $\un{\Fp}$. It was the first cryptosystem to be based on the Diffie-Hellman problem, hence the name of the problem. More generally, it was the first successful implementation of a public-key cryptosystem to be published.

%TODO show somewhere that \un{\Fpn} is actually the group variety xy - 1 = 0, points on this variety are represented as (a, a^-1) for some non-zero a, and the group operation is (x, y) + (x', y') = (xx', yy')
%TODO show somewhere that (\Fpn, +) is actually the trivial group variety, points on this variety are represented as (a) for some a, and the group operation is (x) + (x') = (x + x')

To establish a common secret using this scheme, Alice simply computes the point multiplication $[s_A] K_B$ using her secret key and Bob's public key, and Bob similarly computes $[s_B] K_A$. These computation reduce to $[s_A] K_B = [s_As_B] G$ and $[s_B] K_A = [s_Bs_A] G$, which are clearly equal. Sage code for this procedure is given in \cref{alg:dh}.

\begin{alg}{Diffie-Hellman key-agreement}{dh}
\begin{sagecode}
def diffie_hellman_key_agreement(s_A, K_B):
    """
    s_A: Alice's secret key
    K_B: Bob's public key
    """
    return s_A * K_B
\end{sagecode}
\end{alg}

%TODO i think i should get rid of the kwargs (make params global?), put the keys as the first arguments, and remove tuples from function definitions (use unpacking to chain)

Assuming knowledge of the system parameters and Alice and Bob's public keys $K_A$ and $K_B$, computing the shared secret $[s_A s_B] G$ is clearly equivalent to solving the Diffie-Hellman problem.
